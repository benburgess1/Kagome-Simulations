\chapter{Conclusions}\label{Sec:Conclusions}

This report has presented the results of computational simulations of one and two bosons in a kagome lattice. The single-particle behaviour can be fully understood from the bandstructure of the system. Particles in the dispersive bands of the system are mobile, whereas particles in the flat band are stationary and remain localised in the vicinity of their initial position.

Two-particle systems have been studied across the full range of interaction strengths. For weak interactions, direct products of single particle Bloch states are good two-particle eigenstates, but with eigenvalues shifted by the interaction energy. This lifts the flat band degeneracy, enabling the transport of bound pairs of particles. Particles in spatially separated flat band states have zero interaction energy, and hence remain stationary.

With strong interactions, the system eigenspectrum divides into a doublon band and scattering band. States in the two bands cannot inter-convert due to the prohibitively large interaction energy cost. Systems composed of scattering states behave qualitatively similarly to systems of non-interacting particles, with the difference that the density tends to distribute more evenly throughout the lattice. The doublon behaviour is well-described by an effective single-particle Hamiltonian; the particles are bound together on the same site as they propagate.

This project has highlighted the rich physics that arises from interactions between just two particles in a flat band. It is natural to ask whether systems with more particles can be understood through the microscopic behaviour elucidated in this report, or whether larger, collective descriptions are required. To investigate this question, with greater computing power, similar exact simulations for slightly greater numbers of particles could be performed. Beyond this, approximations, such as mean-field methods, would be required. Accordingly, it is plausible that experiments, in particular with cold atoms, are able to make greater progress in the exploration of flat band systems than numerical research.

Beyond higher particle numbers, other extensions to this work could include considering the effect of the external trapping potential used in cold atom experiments, and a greater variety of initial states. The results presented throughout this report provide a broad, solid foundation on which to develop understanding for these additional cases.