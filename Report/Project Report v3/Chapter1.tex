\pagenumbering{arabic}

\chapter{Introduction}\label{Introduction}

In dispersionless (`flat') Bloch bands, the energy is constant and independent of momentum \cite{Leykam}. The physics of systems with flat bands is therefore dominated by interactions rather than the kinetic energy, leading to various interesting strongly-correlated phenomena. Notable recent topics of research include fractional and anomalous quantum Hall physics \cite{Neupert,Tang,Parameswaran,Leonard,Kida,Liu}, topological states of matter \cite{Rachel,Sun,Bergman,Yin,Li,Kang}, unconventional superconductivity \cite{Kopnin,Sato,Xu,Yankowitz,Mielke,Guguchia}, and other exotic phases \cite{Isakov06,You,Katsura}.

Flat bands arise in tight-binding models on certain lattices, including the dice lattice \cite{Sutherland}, Lieb lattice \cite{Lieb}, and the kagome lattice \cite{Syozi}, the subject of this project. Previous research has predominantly studied the rich equilibrium behaviour of kagome systems (which originates from the Dirac cones in the bandstructure and the frustrated geometry, as well as the flat band). Numerical studies have predicted numerous topological phases \cite{Bergman,Isakov07,Guo,Roychowdhury}, exotic superfluid phases \cite{Isakov06,Huber,You,Julku21,Julku21-2}, and distinctive magnetic properties, including `spin liquids' \cite{Huse,Balents,Plat,Broholm}. Simultaneously, experimental research has observed topological flat bands \cite{Kang,Li}, unconventional superconductivity \cite{Yin,Mielke,Guguchia} and the anomalous Hall effect \cite{Kida,Liu} in kagome materials. There is, however, little research into non-equilibrium, dynamical behaviour in kagome systems, which this project will address. This limited relevant prior research will be detailed in Section \ref{Sec:Theory_and_Literature}.

Beyond this scarcity of existing research, another motivation for this project is the emerging capability of experimentally studying dynamics in kagome systems, through quantum simulation with ultracold atoms \cite{Bloch,Gross}. This experimental field involves creating an optical lattice with lasers, forming an effective potential landscape for (fermionic or bosonic) alkali atoms. These experiments have numerous advantages over traditional condensed matter systems: they are well-described by simple Hamiltonians with controllable parameters; they are defect-free; the absence of dissipation and decoherence mechanisms facilitate investigating non-equilibrium physics \cite{Langen}; and the dynamical timescales are sufficiently long to be resolved \cite{Polkovnikov}.

Although the kagome lattice was first realised in such an experiment in 2012 \cite{Jo}, the flat band is the highest in energy in the kagome bandstructure and therefore difficult to populate. Consequently, previous cold atom flat band research has mainly employed the Lieb lattice \cite{Leykam}. However, techniques for preparing negative-temperature cold atom systems have been demonstrated \cite{Schneider13}. Here, the higher-energy states are more populated, opening the possibility of flat band research using optical kagome lattices. This project complements the ongoing experimental research in this field by my supervisor, Prof. Ulrich Schneider.

In this report, exact numerical simulations of bosonic kagome systems are presented, with a particular focus on the effects of interactions within the flat band. A fundamental, microscopic understanding of the behaviour of one- and two-particle systems is developed for different regimes of interaction strength. Computational limitations precluded studying systems with greater particle numbers, as discussed in Appendix \ref{App:ThreeParticle}. The report is structured as follows. Section \ref{Sec:Theory_and_Literature} sets out the relevant theoretical background and literature, following which the scope and aims of the project can be described in more detail. Section \ref{Sec:Methodology} details the methodology of the computational simulations. Section \ref{Sec:SingleParticleResults} presents single-particle simulations, and Section \ref{Sec:TwoParticleResults} two-particle simulations, with the dynamical behaviour examined both qualitatively and quantitatively. Finally, Section \ref{Sec:Conclusions} gives the conclusions and outlook from the project.